% Options for packages loaded elsewhere
\PassOptionsToPackage{unicode}{hyperref}
\PassOptionsToPackage{hyphens}{url}
%
\documentclass[
]{article}
\usepackage{amsmath,amssymb}
\usepackage{iftex}
\ifPDFTeX
  \usepackage[T1]{fontenc}
  \usepackage[utf8]{inputenc}
  \usepackage{textcomp} % provide euro and other symbols
\else % if luatex or xetex
  \usepackage{unicode-math} % this also loads fontspec
  \defaultfontfeatures{Scale=MatchLowercase}
  \defaultfontfeatures[\rmfamily]{Ligatures=TeX,Scale=1}
\fi
\usepackage{lmodern}
\ifPDFTeX\else
  % xetex/luatex font selection
\fi
% Use upquote if available, for straight quotes in verbatim environments
\IfFileExists{upquote.sty}{\usepackage{upquote}}{}
\IfFileExists{microtype.sty}{% use microtype if available
  \usepackage[]{microtype}
  \UseMicrotypeSet[protrusion]{basicmath} % disable protrusion for tt fonts
}{}
\makeatletter
\@ifundefined{KOMAClassName}{% if non-KOMA class
  \IfFileExists{parskip.sty}{%
    \usepackage{parskip}
  }{% else
    \setlength{\parindent}{0pt}
    \setlength{\parskip}{6pt plus 2pt minus 1pt}}
}{% if KOMA class
  \KOMAoptions{parskip=half}}
\makeatother
\usepackage{xcolor}
\usepackage[margin=1in]{geometry}
\usepackage{color}
\usepackage{fancyvrb}
\newcommand{\VerbBar}{|}
\newcommand{\VERB}{\Verb[commandchars=\\\{\}]}
\DefineVerbatimEnvironment{Highlighting}{Verbatim}{commandchars=\\\{\}}
% Add ',fontsize=\small' for more characters per line
\usepackage{framed}
\definecolor{shadecolor}{RGB}{248,248,248}
\newenvironment{Shaded}{\begin{snugshade}}{\end{snugshade}}
\newcommand{\AlertTok}[1]{\textcolor[rgb]{0.94,0.16,0.16}{#1}}
\newcommand{\AnnotationTok}[1]{\textcolor[rgb]{0.56,0.35,0.01}{\textbf{\textit{#1}}}}
\newcommand{\AttributeTok}[1]{\textcolor[rgb]{0.13,0.29,0.53}{#1}}
\newcommand{\BaseNTok}[1]{\textcolor[rgb]{0.00,0.00,0.81}{#1}}
\newcommand{\BuiltInTok}[1]{#1}
\newcommand{\CharTok}[1]{\textcolor[rgb]{0.31,0.60,0.02}{#1}}
\newcommand{\CommentTok}[1]{\textcolor[rgb]{0.56,0.35,0.01}{\textit{#1}}}
\newcommand{\CommentVarTok}[1]{\textcolor[rgb]{0.56,0.35,0.01}{\textbf{\textit{#1}}}}
\newcommand{\ConstantTok}[1]{\textcolor[rgb]{0.56,0.35,0.01}{#1}}
\newcommand{\ControlFlowTok}[1]{\textcolor[rgb]{0.13,0.29,0.53}{\textbf{#1}}}
\newcommand{\DataTypeTok}[1]{\textcolor[rgb]{0.13,0.29,0.53}{#1}}
\newcommand{\DecValTok}[1]{\textcolor[rgb]{0.00,0.00,0.81}{#1}}
\newcommand{\DocumentationTok}[1]{\textcolor[rgb]{0.56,0.35,0.01}{\textbf{\textit{#1}}}}
\newcommand{\ErrorTok}[1]{\textcolor[rgb]{0.64,0.00,0.00}{\textbf{#1}}}
\newcommand{\ExtensionTok}[1]{#1}
\newcommand{\FloatTok}[1]{\textcolor[rgb]{0.00,0.00,0.81}{#1}}
\newcommand{\FunctionTok}[1]{\textcolor[rgb]{0.13,0.29,0.53}{\textbf{#1}}}
\newcommand{\ImportTok}[1]{#1}
\newcommand{\InformationTok}[1]{\textcolor[rgb]{0.56,0.35,0.01}{\textbf{\textit{#1}}}}
\newcommand{\KeywordTok}[1]{\textcolor[rgb]{0.13,0.29,0.53}{\textbf{#1}}}
\newcommand{\NormalTok}[1]{#1}
\newcommand{\OperatorTok}[1]{\textcolor[rgb]{0.81,0.36,0.00}{\textbf{#1}}}
\newcommand{\OtherTok}[1]{\textcolor[rgb]{0.56,0.35,0.01}{#1}}
\newcommand{\PreprocessorTok}[1]{\textcolor[rgb]{0.56,0.35,0.01}{\textit{#1}}}
\newcommand{\RegionMarkerTok}[1]{#1}
\newcommand{\SpecialCharTok}[1]{\textcolor[rgb]{0.81,0.36,0.00}{\textbf{#1}}}
\newcommand{\SpecialStringTok}[1]{\textcolor[rgb]{0.31,0.60,0.02}{#1}}
\newcommand{\StringTok}[1]{\textcolor[rgb]{0.31,0.60,0.02}{#1}}
\newcommand{\VariableTok}[1]{\textcolor[rgb]{0.00,0.00,0.00}{#1}}
\newcommand{\VerbatimStringTok}[1]{\textcolor[rgb]{0.31,0.60,0.02}{#1}}
\newcommand{\WarningTok}[1]{\textcolor[rgb]{0.56,0.35,0.01}{\textbf{\textit{#1}}}}
\usepackage{graphicx}
\makeatletter
\def\maxwidth{\ifdim\Gin@nat@width>\linewidth\linewidth\else\Gin@nat@width\fi}
\def\maxheight{\ifdim\Gin@nat@height>\textheight\textheight\else\Gin@nat@height\fi}
\makeatother
% Scale images if necessary, so that they will not overflow the page
% margins by default, and it is still possible to overwrite the defaults
% using explicit options in \includegraphics[width, height, ...]{}
\setkeys{Gin}{width=\maxwidth,height=\maxheight,keepaspectratio}
% Set default figure placement to htbp
\makeatletter
\def\fps@figure{htbp}
\makeatother
\setlength{\emergencystretch}{3em} % prevent overfull lines
\providecommand{\tightlist}{%
  \setlength{\itemsep}{0pt}\setlength{\parskip}{0pt}}
\setcounter{secnumdepth}{-\maxdimen} % remove section numbering
\ifLuaTeX
  \usepackage{selnolig}  % disable illegal ligatures
\fi
\usepackage{bookmark}
\IfFileExists{xurl.sty}{\usepackage{xurl}}{} % add URL line breaks if available
\urlstyle{same}
\hypersetup{
  pdftitle={Foundations for statistical inference - Sampling distributions},
  pdfauthor={Emmanuel Kasigazi},
  hidelinks,
  pdfcreator={LaTeX via pandoc}}

\title{Foundations for statistical inference - Sampling distributions}
\author{Emmanuel Kasigazi}
\date{}

\begin{document}
\maketitle

In this lab, you will investigate the ways in which the statistics from
a random sample of data can serve as point estimates for population
parameters. We're interested in formulating a \emph{sampling
distribution} of our estimate in order to learn about the properties of
the estimate, such as its distribution.

\phantomsection\label{boxedtext}
\textbf{Setting a seed:} We will take some random samples and build
sampling distributions in this lab, which means you should set a seed at
the start of your lab. If this concept is new to you, review the lab on
probability.

\subsection{Getting Started}\label{getting-started}

\subsubsection{Load packages}\label{load-packages}

In this lab, we will explore and visualize the data using the
\textbf{tidyverse} suite of packages. We will also use the
\textbf{infer} package for resampling.

Let's load the packages.

\begin{Shaded}
\begin{Highlighting}[]
\FunctionTok{library}\NormalTok{(tidyverse)}
\FunctionTok{library}\NormalTok{(openintro)}
\FunctionTok{library}\NormalTok{(infer)}
\FunctionTok{library}\NormalTok{(tidyverse)}
\FunctionTok{library}\NormalTok{(openintro)}
\FunctionTok{library}\NormalTok{(tidyverse)}
\FunctionTok{library}\NormalTok{(openintro)}
\FunctionTok{library}\NormalTok{(dplyr)}
\FunctionTok{library}\NormalTok{(tinytex)}
\end{Highlighting}
\end{Shaded}

\begin{Shaded}
\begin{Highlighting}[]
\FunctionTok{library}\NormalTok{(tidyverse)}
\FunctionTok{library}\NormalTok{(openintro)}
\end{Highlighting}
\end{Shaded}

\subsubsection{The data}\label{the-data}

A 2019 Gallup report states the following:

\begin{quote}
The premise that scientific progress benefits people has been embodied
in discoveries throughout the ages -- from the development of
vaccinations to the explosion of technology in the past few decades,
resulting in billions of supercomputers now resting in the hands and
pockets of people worldwide. Still, not everyone around the world feels
science benefits them personally.

\textbf{Source:}
\href{https://news.gallup.com/opinion/gallup/268121/world-science-day-knowledge-power.aspx}{World
Science Day: Is Knowledge Power?}
\end{quote}

The Wellcome Global Monitor finds that 20\% of people globally do not
believe that the work scientists do benefits people like them. In this
lab, you will assume this 20\% is a true population proportion and learn
about how sample proportions can vary from sample to sample by taking
smaller samples from the population. We will first create our population
assuming a population size of 100,000. This means 20,000 (20\%) of the
population think the work scientists do does not benefit them personally
and the remaining 80,000 think it does.

\begin{Shaded}
\begin{Highlighting}[]
\NormalTok{global\_monitor }\OtherTok{\textless{}{-}} \FunctionTok{tibble}\NormalTok{(}
  \AttributeTok{scientist\_work =} \FunctionTok{c}\NormalTok{(}\FunctionTok{rep}\NormalTok{(}\StringTok{"Benefits"}\NormalTok{, }\DecValTok{80000}\NormalTok{), }\FunctionTok{rep}\NormalTok{(}\StringTok{"Doesn\textquotesingle{}t benefit"}\NormalTok{, }\DecValTok{20000}\NormalTok{))}
\NormalTok{)}
\end{Highlighting}
\end{Shaded}

The name of the data frame is \texttt{global\_monitor} and the name of
the variable that contains responses to the question \emph{``Do you
believe that the work scientists do benefit people like you?''} is
\texttt{scientist\_work}.

We can quickly visualize the distribution of these responses using a bar
plot.

\begin{Shaded}
\begin{Highlighting}[]
\FunctionTok{ggplot}\NormalTok{(global\_monitor, }\FunctionTok{aes}\NormalTok{(}\AttributeTok{x =}\NormalTok{ scientist\_work)) }\SpecialCharTok{+}
  \FunctionTok{geom\_bar}\NormalTok{() }\SpecialCharTok{+}
  \FunctionTok{labs}\NormalTok{(}
    \AttributeTok{x =} \StringTok{""}\NormalTok{, }\AttributeTok{y =} \StringTok{""}\NormalTok{,}
    \AttributeTok{title =} \StringTok{"Do you believe that the work scientists do benefit people like you?"}
\NormalTok{  ) }\SpecialCharTok{+}
  \FunctionTok{coord\_flip}\NormalTok{() }
\end{Highlighting}
\end{Shaded}

\includegraphics{Part1_sampling_distributions-EKO_files/figure-latex/bar-plot-pop1-1.pdf}

We can also obtain summary statistics to confirm we constructed the data
frame correctly.

\begin{Shaded}
\begin{Highlighting}[]
\NormalTok{global\_monitor }\SpecialCharTok{\%\textgreater{}\%}
  \FunctionTok{count}\NormalTok{(scientist\_work) }\SpecialCharTok{\%\textgreater{}\%}
  \FunctionTok{mutate}\NormalTok{(}\AttributeTok{p =}\NormalTok{ n }\SpecialCharTok{/}\FunctionTok{sum}\NormalTok{(n))}
\end{Highlighting}
\end{Shaded}

\begin{verbatim}
## # A tibble: 2 x 3
##   scientist_work      n     p
##   <chr>           <int> <dbl>
## 1 Benefits        80000   0.8
## 2 Doesn't benefit 20000   0.2
\end{verbatim}

\subsection{The unknown sampling
distribution}\label{the-unknown-sampling-distribution}

In this lab, you have access to the entire population, but this is
rarely the case in real life. Gathering information on an entire
population is often extremely costly or impossible. Because of this, we
often take a sample of the population and use that to understand the
properties of the population.

If you are interested in estimating the proportion of people who don't
think the work scientists do benefits them, you can use the
\texttt{sample\_n} command to survey the population.

\begin{Shaded}
\begin{Highlighting}[]
\NormalTok{samp1 }\OtherTok{\textless{}{-}}\NormalTok{ global\_monitor }\SpecialCharTok{\%\textgreater{}\%}
  \FunctionTok{sample\_n}\NormalTok{(}\DecValTok{50}\NormalTok{)}
\end{Highlighting}
\end{Shaded}

This command collects a simple random sample of size 50 from the
\texttt{global\_monitor} dataset, and assigns the result to
\texttt{samp1}. This is similar to randomly drawing names from a hat
that contains the names of all in the population. Working with these 50
names is considerably simpler than working with all 100,000 people in
the population.

\begin{enumerate}
\def\labelenumi{\arabic{enumi}.}
\tightlist
\item
  Describe the distribution of responses in this sample. How does it
  compare to the distribution of responses in the population.
  \textbf{Hint:} Although the \texttt{sample\_n} function takes a random
  sample of observations (i.e.~rows) from the dataset, you can still
  refer to the variables in the dataset with the same names. Code you
  presented earlier for visualizing and summarizing the population data
  will still be useful for the sample, however be careful to not label
  your proportion \texttt{p} since you're now calculating a sample
  statistic, not a population parameters. You can customize the label of
  the statistics to indicate that it comes from the sample.
\end{enumerate}

\begin{Shaded}
\begin{Highlighting}[]
\NormalTok{samp1 }\SpecialCharTok{\%\textgreater{}\%}
  \FunctionTok{count}\NormalTok{(scientist\_work) }\SpecialCharTok{\%\textgreater{}\%}
  \FunctionTok{mutate}\NormalTok{(}\AttributeTok{sample =}\NormalTok{ n }\SpecialCharTok{/}\FunctionTok{sum}\NormalTok{(n))}
\end{Highlighting}
\end{Shaded}

\begin{verbatim}
## # A tibble: 2 x 3
##   scientist_work      n sample
##   <chr>           <int>  <dbl>
## 1 Benefits           39   0.78
## 2 Doesn't benefit    11   0.22
\end{verbatim}

scientist\_work n sample

1 Benefits 40 0.8

2 Doesn't benefit 10 0.2

\begin{Shaded}
\begin{Highlighting}[]
\FunctionTok{ggplot}\NormalTok{(samp1, }\FunctionTok{aes}\NormalTok{(}\AttributeTok{x =}\NormalTok{ scientist\_work)) }\SpecialCharTok{+}
  \FunctionTok{geom\_bar}\NormalTok{() }\SpecialCharTok{+}
  \FunctionTok{labs}\NormalTok{(}
    \AttributeTok{x =} \StringTok{""}\NormalTok{, }\AttributeTok{y =} \StringTok{""}\NormalTok{,}
    \AttributeTok{title =} \StringTok{"(Sample)Do you believe that the work scientists do benefit people like you?"}
\NormalTok{  ) }\SpecialCharTok{+}
  \FunctionTok{coord\_flip}\NormalTok{() }
\end{Highlighting}
\end{Shaded}

\includegraphics{Part1_sampling_distributions-EKO_files/figure-latex/bar-plot-pop2-1.pdf}

\textbf{In comparison it seems like its the same to me}

If you're interested in estimating the proportion of all people who do
not believe that the work scientists do benefits them, but you do not
have access to the population data, your best single guess is the sample
mean.

\begin{Shaded}
\begin{Highlighting}[]
\NormalTok{samp1 }\SpecialCharTok{\%\textgreater{}\%}
  \FunctionTok{count}\NormalTok{(scientist\_work) }\SpecialCharTok{\%\textgreater{}\%}
  \FunctionTok{mutate}\NormalTok{(}\AttributeTok{p\_hat =}\NormalTok{ n }\SpecialCharTok{/}\FunctionTok{sum}\NormalTok{(n))}
\end{Highlighting}
\end{Shaded}

\begin{verbatim}
## # A tibble: 2 x 3
##   scientist_work      n p_hat
##   <chr>           <int> <dbl>
## 1 Benefits           39  0.78
## 2 Doesn't benefit    11  0.22
\end{verbatim}

Depending on which 50 people you selected, your estimate could be a bit
above or a bit below the true population proportion of 0.22. In general,
though, the sample proportion turns out to be a pretty good estimate of
the true population proportion, and you were able to get it by sampling
less than 1\% of the population.

\begin{enumerate}
\def\labelenumi{\arabic{enumi}.}
\setcounter{enumi}{1}
\item
  Would you expect the sample proportion to match the sample proportion
  of another student's sample? Why, or why not? If the answer is no,
  would you expect the proportions to be somewhat different or very
  different? Ask a student team to confirm your answer.

  \textbf{Yes, because samples usually tend to follow the population:
  There might be one or two oint difference but generally I expect the
  sample to follow / match mine}
\end{enumerate}

\textbf{Insert your answer here}

\begin{enumerate}
\def\labelenumi{\arabic{enumi}.}
\setcounter{enumi}{2}
\tightlist
\item
  Take a second sample, also of size 50, and call it \texttt{samp2}. How
  does the sample proportion of \texttt{samp2} compare with that of
  \texttt{samp1}? Suppose we took two more samples, one of size 100 and
  one of size 1000. Which would you think would provide a more accurate
  estimate of the population proportion?
\end{enumerate}

\begin{Shaded}
\begin{Highlighting}[]
\NormalTok{samp2 }\OtherTok{\textless{}{-}}\NormalTok{ global\_monitor }\SpecialCharTok{\%\textgreater{}\%}
  \FunctionTok{sample\_n}\NormalTok{(}\DecValTok{50}\NormalTok{)}
\end{Highlighting}
\end{Shaded}

\begin{Shaded}
\begin{Highlighting}[]
\NormalTok{samp2 }\SpecialCharTok{\%\textgreater{}\%}
  \FunctionTok{count}\NormalTok{(scientist\_work) }\SpecialCharTok{\%\textgreater{}\%}
  \FunctionTok{mutate}\NormalTok{(}\AttributeTok{sample2 =}\NormalTok{ n }\SpecialCharTok{/}\FunctionTok{sum}\NormalTok{(n))}
\end{Highlighting}
\end{Shaded}

\begin{verbatim}
## # A tibble: 2 x 3
##   scientist_work      n sample2
##   <chr>           <int>   <dbl>
## 1 Benefits           40     0.8
## 2 Doesn't benefit    10     0.2
\end{verbatim}

\begin{verbatim}
  scientist_work      n sample2
\end{verbatim}

1 Benefits 39 0.78

2 Doesn't benefit 11 0.22

\begin{Shaded}
\begin{Highlighting}[]
\FunctionTok{ggplot}\NormalTok{(samp2, }\FunctionTok{aes}\NormalTok{(}\AttributeTok{x =}\NormalTok{ scientist\_work)) }\SpecialCharTok{+}
  \FunctionTok{geom\_bar}\NormalTok{() }\SpecialCharTok{+}
  \FunctionTok{labs}\NormalTok{(}
    \AttributeTok{x =} \StringTok{""}\NormalTok{, }\AttributeTok{y =} \StringTok{""}\NormalTok{,}
    \AttributeTok{title =} \StringTok{"(Sample)Do you believe that the work scientists do benefit people like you?"}
\NormalTok{  ) }\SpecialCharTok{+}
  \FunctionTok{coord\_flip}\NormalTok{() }
\end{Highlighting}
\end{Shaded}

\includegraphics{Part1_sampling_distributions-EKO_files/figure-latex/bar-plot-pop3-1.pdf}

\textbf{The second sample is almost similar to the first, with just a
few points of difference, which don't really make much of an impact.
While the first sample was 0.8, indicating some benefits, the second
sample is 0.78, which is pretty close. A sample of 1,000 would provide a
more accurate depiction because it better represents and is closer to
the population.}

Not surprisingly, every time you take another random sample, you might
get a different sample proportion. It's useful to get a sense of just
how much variability you should expect when estimating the population
mean this way. The distribution of sample proportions, called the
\emph{sampling distribution (of the proportion)}, can help you
understand this variability. In this lab, because you have access to the
population, you can build up the sampling distribution for the sample
proportion by repeating the above steps many times. Here, we use R to
take 15,000 different samples of size 50 from the population, calculate
the proportion of responses in each sample, filter for only the
\emph{Doesn't benefit} responses, and store each result in a vector
called \texttt{sample\_props50}. Note that we specify that
\texttt{replace\ =\ TRUE} since sampling distributions are constructed
by sampling with replacement.

\begin{Shaded}
\begin{Highlighting}[]
\NormalTok{sample\_props50 }\OtherTok{\textless{}{-}}\NormalTok{ global\_monitor }\SpecialCharTok{\%\textgreater{}\%}
                    \FunctionTok{rep\_sample\_n}\NormalTok{(}\AttributeTok{size =} \DecValTok{50}\NormalTok{, }\AttributeTok{reps =} \DecValTok{15000}\NormalTok{, }\AttributeTok{replace =} \ConstantTok{TRUE}\NormalTok{) }\SpecialCharTok{\%\textgreater{}\%}
                    \FunctionTok{count}\NormalTok{(scientist\_work) }\SpecialCharTok{\%\textgreater{}\%}
                    \FunctionTok{mutate}\NormalTok{(}\AttributeTok{p\_hat =}\NormalTok{ n }\SpecialCharTok{/}\FunctionTok{sum}\NormalTok{(n)) }\SpecialCharTok{\%\textgreater{}\%}
                    \FunctionTok{filter}\NormalTok{(scientist\_work }\SpecialCharTok{==} \StringTok{"Doesn\textquotesingle{}t benefit"}\NormalTok{)}
\end{Highlighting}
\end{Shaded}

And we can visualize the distribution of these proportions with a
histogram.

\begin{Shaded}
\begin{Highlighting}[]
\FunctionTok{ggplot}\NormalTok{(}\AttributeTok{data =}\NormalTok{ sample\_props50, }\FunctionTok{aes}\NormalTok{(}\AttributeTok{x =}\NormalTok{ p\_hat)) }\SpecialCharTok{+}
  \FunctionTok{geom\_histogram}\NormalTok{(}\AttributeTok{binwidth =} \FloatTok{0.02}\NormalTok{) }\SpecialCharTok{+}
  \FunctionTok{labs}\NormalTok{(}
    \AttributeTok{x =} \StringTok{"p\_hat (Doesn\textquotesingle{}t benefit)"}\NormalTok{,}
    \AttributeTok{title =} \StringTok{"Sampling distribution of p\_hat"}\NormalTok{,}
    \AttributeTok{subtitle =} \StringTok{"Sample size = 50, Number of samples = 15000"}
\NormalTok{  )}
\end{Highlighting}
\end{Shaded}

Next, you will review how this set of code works.

\begin{enumerate}
\def\labelenumi{\arabic{enumi}.}
\setcounter{enumi}{3}
\item
  How many elements are there in \texttt{sample\_props50}? Describe the
  sampling distribution, and be sure to specifically note its center.
  Make sure to include a plot of the distribution in your answer.

  \textbf{The dataset \texttt{sample\_props50} contains 15,000 elements.
  Each element represents a sample proportion calculated from a sample
  of size 50 taken from the population, specifically for responses
  labeled ``Doesn't benefit''.}

  \textbf{Distribution characteristics:}

  \textbf{-Shape: The distribution appears approximately normal
  (bell-shaped) a normal distribution, which aligns with the Central
  Limit Theorem (CLT). Even though individual samples may vary, when we
  take many samples and compute their proportions, the distribution of
  those sample proportions approximates a normal curve.}

  \textbf{-Center: The peak of the distribution appears to be around 0.2
  (or 20\%) suggesting that the mean of \texttt{p\_hat} is close to
  0.2.}

  \textbf{-Spread (Standard Error, SE): The variability in the histogram
  gives insight into the sampling variability. The standard deviation of
  this sampling distribution (Standard Error, SE) can be estimated using
  the formula:}

  SE=p(1−p)nSE =
  \sqrt{\frac{p(1 - p)}{n}}SE\textbf{=}np\textbf{(1−}p\textbf{)\hspace{0pt}\hspace{0pt}}

  \textbf{where} p≈0.2p \approx 0.2p\textbf{≈0.2 (from the histogram)
  and} n=50n = 50n\textbf{=50.}
\end{enumerate}

\textbf{The sampling distribution demonstrates the variability we would
expect to see when taking repeated samples of size 50 from the
population. The fact that it's approximately normal is consistent with
the Central Limit Theorem, which states that the sampling distribution
of a proportion will be approximately normal for sufficiently large
sample sizes.}

\subsection{Interlude: Sampling
distributions}\label{interlude-sampling-distributions}

The idea behind the \texttt{rep\_sample\_n} function is
\emph{repetition}. Earlier, you took a single sample of size \texttt{n}
(50) from the population of all people in the population. With this new
function, you can repeat this sampling procedure \texttt{rep} times in
order to build a distribution of a series of sample statistics, which is
called the \textbf{sampling distribution}.

Note that in practice one rarely gets to build true sampling
distributions, because one rarely has access to data from the entire
population.

Without the \texttt{rep\_sample\_n} function, this would be painful. We
would have to manually run the following code 15,000 times

\begin{Shaded}
\begin{Highlighting}[]
\NormalTok{global\_monitor }\SpecialCharTok{\%\textgreater{}\%}
  \FunctionTok{sample\_n}\NormalTok{(}\AttributeTok{size =} \DecValTok{50}\NormalTok{, }\AttributeTok{replace =} \ConstantTok{TRUE}\NormalTok{) }\SpecialCharTok{\%\textgreater{}\%}
  \FunctionTok{count}\NormalTok{(scientist\_work) }\SpecialCharTok{\%\textgreater{}\%}
  \FunctionTok{mutate}\NormalTok{(}\AttributeTok{p\_hat =}\NormalTok{ n }\SpecialCharTok{/}\FunctionTok{sum}\NormalTok{(n)) }\SpecialCharTok{\%\textgreater{}\%}
  \FunctionTok{filter}\NormalTok{(scientist\_work }\SpecialCharTok{==} \StringTok{"Doesn\textquotesingle{}t benefit"}\NormalTok{)}
\end{Highlighting}
\end{Shaded}

\begin{verbatim}
## # A tibble: 1 x 3
##   scientist_work      n p_hat
##   <chr>           <int> <dbl>
## 1 Doesn't benefit     6  0.12
\end{verbatim}

as well as store the resulting sample proportions each time in a
separate vector.

Note that for each of the 15,000 times we computed a proportion, we did
so from a \textbf{different} sample!

\begin{enumerate}
\def\labelenumi{\arabic{enumi}.}
\setcounter{enumi}{4}
\tightlist
\item
  To make sure you understand how sampling distributions are built, and
  exactly what the \texttt{rep\_sample\_n} function does, try modifying
  the code to create a sampling distribution of \textbf{25 sample
  proportions} from \textbf{samples of size 10}, and put them in a data
  frame named \texttt{sample\_props\_small}. Print the output. How many
  observations are there in this object called
  \texttt{sample\_props\_small}? What does each observation represent?
\end{enumerate}

\begin{Shaded}
\begin{Highlighting}[]
\NormalTok{sample\_props\_small }\OtherTok{\textless{}{-}}\NormalTok{ global\_monitor }\SpecialCharTok{\%\textgreater{}\%}
                    \FunctionTok{rep\_sample\_n}\NormalTok{(}\AttributeTok{size =} \DecValTok{10}\NormalTok{, }\AttributeTok{reps =} \DecValTok{25}\NormalTok{, }\AttributeTok{replace =} \ConstantTok{TRUE}\NormalTok{) }\SpecialCharTok{\%\textgreater{}\%}
                    \FunctionTok{count}\NormalTok{(scientist\_work) }\SpecialCharTok{\%\textgreater{}\%}
                    \FunctionTok{mutate}\NormalTok{(}\AttributeTok{p\_hat =}\NormalTok{ n }\SpecialCharTok{/}\FunctionTok{sum}\NormalTok{(n)) }\SpecialCharTok{\%\textgreater{}\%}
                    \FunctionTok{filter}\NormalTok{(scientist\_work }\SpecialCharTok{==} \StringTok{"Doesn\textquotesingle{}t benefit"}\NormalTok{)}
\end{Highlighting}
\end{Shaded}

\begin{Shaded}
\begin{Highlighting}[]
\NormalTok{sample\_props\_small}
\end{Highlighting}
\end{Shaded}

\begin{verbatim}
## # A tibble: 22 x 4
## # Groups:   replicate [22]
##    replicate scientist_work      n p_hat
##        <int> <chr>           <int> <dbl>
##  1         1 Doesn't benefit     3   0.3
##  2         2 Doesn't benefit     3   0.3
##  3         3 Doesn't benefit     4   0.4
##  4         4 Doesn't benefit     2   0.2
##  5         5 Doesn't benefit     1   0.1
##  6         8 Doesn't benefit     3   0.3
##  7         9 Doesn't benefit     4   0.4
##  8        10 Doesn't benefit     2   0.2
##  9        11 Doesn't benefit     1   0.1
## 10        12 Doesn't benefit     3   0.3
## # i 12 more rows
\end{verbatim}

\begin{Shaded}
\begin{Highlighting}[]
\FunctionTok{ggplot}\NormalTok{(}\AttributeTok{data =}\NormalTok{ sample\_props\_small, }\FunctionTok{aes}\NormalTok{(}\AttributeTok{x =}\NormalTok{ p\_hat)) }\SpecialCharTok{+}
  \FunctionTok{geom\_histogram}\NormalTok{(}\AttributeTok{binwidth =} \FloatTok{0.05}\NormalTok{) }\SpecialCharTok{+}
  \FunctionTok{labs}\NormalTok{(}
    \AttributeTok{x =} \StringTok{"p\_hat (Doesn\textquotesingle{}t benefit)"}\NormalTok{,}
    \AttributeTok{title =} \StringTok{"Sampling distribution of p\_hat"}\NormalTok{,}
    \AttributeTok{subtitle =} \StringTok{"Sample size = 10, Number of samples = 25"}
\NormalTok{  )}
\end{Highlighting}
\end{Shaded}

\includegraphics{Part1_sampling_distributions-EKO_files/figure-latex/visualizing the smaller sampling distribution-1.pdf}

\textbf{Looking at sample\_props\_small, there should be 25 observations
in total, because we specified reps = 25 in the code. Each observation
in sample\_props\_small represents one sample's results, specifically:}

\textbf{-It represents the proportion of ``Doesn't benefit'' responses
(p\_hat) from a single random sample of 10 people from the population}

\textbf{-Each p\_hat was calculated by taking the number of ``Doesn't
benefit'' responses in that sample of 10 people and dividing by 10}

\textbf{So each row in sample\_props\_small tells us: out of one random
sample of 10 people, what proportion said scientists' work ``Doesn't
benefit'' society. We did this sampling process 25 different times,
which is why we have 25 observations.}

\subsection{Sample size and the sampling
distribution}\label{sample-size-and-the-sampling-distribution}

Mechanics aside, let's return to the reason we used the
\texttt{rep\_sample\_n} function: to compute a sampling distribution,
specifically, the sampling distribution of the proportions from samples
of 50 people.

\begin{Shaded}
\begin{Highlighting}[]
\FunctionTok{ggplot}\NormalTok{(}\AttributeTok{data =}\NormalTok{ sample\_props50, }\FunctionTok{aes}\NormalTok{(}\AttributeTok{x =}\NormalTok{ p\_hat)) }\SpecialCharTok{+}
  \FunctionTok{geom\_histogram}\NormalTok{(}\AttributeTok{binwidth =} \FloatTok{0.02}\NormalTok{)}
\end{Highlighting}
\end{Shaded}

The sampling distribution that you computed tells you much about
estimating the true proportion of people who think that the work
scientists do doesn't benefit them. Because the sample proportion is an
unbiased estimator, the sampling distribution is centered at the true
population proportion, and the spread of the distribution indicates how
much variability is incurred by sampling only 50 people at a time from
the population.

In the remainder of this section, you will work on getting a sense of
the effect that sample size has on your sampling distribution.

\begin{enumerate}
\def\labelenumi{\arabic{enumi}.}
\setcounter{enumi}{5}
\tightlist
\item
  Use the app below to create sampling distributions of proportions of
  \emph{Doesn't benefit} from samples of size 10, 50, and 100. Use 5,000
  simulations. What does each observation in the sampling distribution
  represent? How does the mean, standard error, and shape of the
  sampling distribution change as the sample size increases? How (if at
  all) do these values change if you increase the number of simulations?
  (You do not need to include plots in your answer.)
\end{enumerate}

\textbf{What each observation represents:}

Each observation in these sampling distributions represents one sample
proportion (p\_hat) of people who responded ``Doesn't benefit'' from a
single random sample For sample size 10, each p\_hat is calculated from
10 people For sample size 50, each p\_hat is calculated from 50 people
For sample size 100, each p\_hat is calculated from 100 people

Changes in the sampling distribution as sample size increases:

\textbf{Mean:}

The mean appears to stay relatively constant around 0.15-0.20 across all
sample sizes. This is expected since the mean of the sampling
distribution should equal the true population proportion

\textbf{Standard Error (spread of the distribution):}

The spread clearly decreases as sample size increases. Sample size 10:
Widest spread, with proportions varying greatly. Sample size 50:
Noticeably tighter spread. Sample size 100: Tightest spread This follows
the rule that standard error decreases with the square root of the
sample size

\textbf{Shape:}

Sample size 10: Discrete jumps, less normal-looking. Sample size 50:
More bell-shaped, approaching normal. Sample size 100: Most clearly
normal. This illustrates the Central Limit Theorem - as sample size
increases, the sampling distribution becomes more normal

\textbf{Effect of increasing number of simulations (currently 5000):}

\textbf{Mean:} Stays the same because the underlying population
proportion does not change. \textbf{Standard Error:} Does not change
because SE depends on \textbf{sample size}, not the number of
simulations.\\
\textbf{Shape:} The histogram becomes \textbf{smoother} as the number of
simulations increases because more observations provide a clearer
picture of the true distribution.\\
More simulations would give a more precise picture of the sampling
distribution but wouldn't fundamentally change its characteristics. The
``jumpy'' nature of the n=10 histogram would become smoother but would
still show discrete possible values

\begin{center}\rule{0.5\linewidth}{0.5pt}\end{center}

\subsection{More Practice}\label{more-practice}

So far, you have only focused on estimating the proportion of those you
think the work scientists doesn't benefit them. Now, you'll try to
estimate the proportion of those who think it does.

Note that while you might be able to answer some of these questions
using the app, you are expected to write the required code and produce
the necessary plots and summary statistics. You are welcome to use the
app for exploration.

\begin{enumerate}
\def\labelenumi{\arabic{enumi}.}
\setcounter{enumi}{6}
\tightlist
\item
  Take a sample of size 15 from the population and calculate the
  proportion of people in this sample who think the work scientists do
  enhances their lives. Using this sample, what is your best point
  estimate of the population proportion of people who think the work
  scientists do enchances their lives?
\end{enumerate}

\begin{Shaded}
\begin{Highlighting}[]
\CommentTok{\# Take a sample of size 15}
\NormalTok{samp\_15 }\OtherTok{\textless{}{-}}\NormalTok{ global\_monitor }\SpecialCharTok{\%\textgreater{}\%}
  \FunctionTok{sample\_n}\NormalTok{(}\DecValTok{15}\NormalTok{)}

\CommentTok{\# Calculate proportion who think science benefits them}
\NormalTok{samp\_15 }\SpecialCharTok{\%\textgreater{}\%}
  \FunctionTok{count}\NormalTok{(scientist\_work) }\SpecialCharTok{\%\textgreater{}\%}
  \FunctionTok{mutate}\NormalTok{(}\AttributeTok{p\_hat =}\NormalTok{ n }\SpecialCharTok{/}\FunctionTok{sum}\NormalTok{(n)) }\SpecialCharTok{\%\textgreater{}\%}
  \FunctionTok{filter}\NormalTok{(scientist\_work }\SpecialCharTok{==} \StringTok{"Benefits"}\NormalTok{)}
\end{Highlighting}
\end{Shaded}

\begin{verbatim}
## # A tibble: 1 x 3
##   scientist_work     n p_hat
##   <chr>          <int> <dbl>
## 1 Benefits          14 0.933
\end{verbatim}

\begin{Shaded}
\begin{Highlighting}[]
\CommentTok{\# Visualize the sample distribution}
\FunctionTok{ggplot}\NormalTok{(samp\_15, }\FunctionTok{aes}\NormalTok{(}\AttributeTok{x =}\NormalTok{ scientist\_work)) }\SpecialCharTok{+}
  \FunctionTok{geom\_bar}\NormalTok{() }\SpecialCharTok{+}
  \FunctionTok{labs}\NormalTok{(}
    \AttributeTok{x =} \StringTok{""}\NormalTok{, }\AttributeTok{y =} \StringTok{""}\NormalTok{,}
    \AttributeTok{title =} \StringTok{"Do you believe that the work scientists do benefit people like you?"}\NormalTok{,}
    \AttributeTok{subtitle =} \StringTok{"Sample size = 15"}
\NormalTok{  ) }\SpecialCharTok{+}
  \FunctionTok{coord\_flip}\NormalTok{()}
\end{Highlighting}
\end{Shaded}

\includegraphics{Part1_sampling_distributions-EKO_files/figure-latex/visualitzing-1.pdf}

\textbf{The best point estimate for the population proportion of people
who think scientists' work ``Benefits'' them would be the sample
proportion. From the plot:}

\textbf{About 12 people responded ``Benefits'' out of the 15 people
sampled Therefore, p̂ = 12/15 = 0.80 or 80\%}

\textbf{This sample proportion (0.80) is our best point estimate of the
true population proportion. Interestingly, this matches exactly with the
true population proportion that was used to create the original dataset
(80,000/100,000 = 0.80 or 80\%), though in real-world sampling we
wouldn't typically know the true population value. While this is just
one sample and could vary from the true population proportion due to
sampling variability, it represents our best single guess at the
population parameter based on the available sample data}

\begin{enumerate}
\def\labelenumi{\arabic{enumi}.}
\setcounter{enumi}{7}
\tightlist
\item
  Since you have access to the population, simulate the sampling
  distribution of proportion of those who think the work scientists do
  enchances their lives for samples of size 15 by taking 2000 samples
  from the population of size 15 and computing 2000 sample proportions.
  Store these proportions in as \texttt{sample\_props15}. Plot the data,
  then describe the shape of this sampling distribution. Based on this
  sampling distribution, what would you guess the true proportion of
  those who think the work scientists do enchances their lives to be?
  Finally, calculate and report the population proportion.
\end{enumerate}

\begin{Shaded}
\begin{Highlighting}[]
\CommentTok{\# Generate sampling distribution with 2000 samples of size 15}
\NormalTok{sample\_props15 }\OtherTok{\textless{}{-}}\NormalTok{ global\_monitor }\SpecialCharTok{\%\textgreater{}\%}
                    \FunctionTok{rep\_sample\_n}\NormalTok{(}\AttributeTok{size =} \DecValTok{15}\NormalTok{, }\AttributeTok{reps =} \DecValTok{2000}\NormalTok{, }\AttributeTok{replace =} \ConstantTok{TRUE}\NormalTok{) }\SpecialCharTok{\%\textgreater{}\%}
                    \FunctionTok{count}\NormalTok{(scientist\_work) }\SpecialCharTok{\%\textgreater{}\%}
                    \FunctionTok{mutate}\NormalTok{(}\AttributeTok{p\_hat =}\NormalTok{ n }\SpecialCharTok{/}\FunctionTok{sum}\NormalTok{(n)) }\SpecialCharTok{\%\textgreater{}\%}
                    \FunctionTok{filter}\NormalTok{(scientist\_work }\SpecialCharTok{==} \StringTok{"Benefits"}\NormalTok{)}
\end{Highlighting}
\end{Shaded}

\begin{Shaded}
\begin{Highlighting}[]
\CommentTok{\# Create histogram of the sampling distribution}
\FunctionTok{ggplot}\NormalTok{(}\AttributeTok{data =}\NormalTok{ sample\_props15, }\FunctionTok{aes}\NormalTok{(}\AttributeTok{x =}\NormalTok{ p\_hat)) }\SpecialCharTok{+}
  \FunctionTok{geom\_histogram}\NormalTok{(}\AttributeTok{binwidth =} \FloatTok{0.05}\NormalTok{) }\SpecialCharTok{+}
  \FunctionTok{labs}\NormalTok{(}
    \AttributeTok{x =} \StringTok{"p\_hat (Benefits)"}\NormalTok{,}
    \AttributeTok{title =} \StringTok{"Sampling distribution of p\_hat"}\NormalTok{,}
    \AttributeTok{subtitle =} \StringTok{"Sample size = 15, Number of samples = 2000"}
\NormalTok{  )}
\end{Highlighting}
\end{Shaded}

\includegraphics{Part1_sampling_distributions-EKO_files/figure-latex/visualtion-1.pdf}

\textbf{The distribution is roughly bell-shaped but appears slightly
left-skewed The peak is around 0.8-0.9 (80-90\%) There's some natural
sampling variability, with proportions ranging from about 0.3 to 0.9
Most of the sample proportions are concentrated between 0.7 and 0.9}

\textbf{Best guess for true proportion: Based on this sampling
distribution, I would guess the true population proportion is around 0.8
(80\%) because:}

\textbf{This value is near the center of the distribution. The highest
concentration of sample proportions occurs around this value. The
sampling distribution should be centered at or near the true population
proportion}

\textbf{Actual population proportion}:

\begin{Shaded}
\begin{Highlighting}[]
\NormalTok{global\_monitor }\SpecialCharTok{\%\textgreater{}\%}
  \FunctionTok{count}\NormalTok{(scientist\_work) }\SpecialCharTok{\%\textgreater{}\%}
  \FunctionTok{mutate}\NormalTok{(}\AttributeTok{p =}\NormalTok{ n }\SpecialCharTok{/}\FunctionTok{sum}\NormalTok{(n)) }\SpecialCharTok{\%\textgreater{}\%}
  \FunctionTok{filter}\NormalTok{(scientist\_work }\SpecialCharTok{==} \StringTok{"Benefits"}\NormalTok{)}
\end{Highlighting}
\end{Shaded}

\begin{verbatim}
## # A tibble: 1 x 3
##   scientist_work     n     p
##   <chr>          <int> <dbl>
## 1 Benefits       80000   0.8
\end{verbatim}

\textbf{The true population proportion is 0.80 (80,000/100,000 = 0.8),
which aligns well with what the sampling distribution suggested. This
demonstrates how the sampling distribution of sample proportions can
give us good insight into the true population parameter, even with a
relatively small sample size of 15.}

\begin{enumerate}
\def\labelenumi{\arabic{enumi}.}
\setcounter{enumi}{8}
\tightlist
\item
  Change your sample size from 15 to 150, then compute the sampling
  distribution using the same method as above, and store these
  proportions in a new object called \texttt{sample\_props150}. Describe
  the shape of this sampling distribution and compare it to the sampling
  distribution for a sample size of 15. Based on this sampling
  distribution, what would you guess to be the true proportion of those
  who think the work scientists do enchances their lives?
\end{enumerate}

\begin{Shaded}
\begin{Highlighting}[]
\CommentTok{\# Generate sampling distribution with 2000 samples of size 150}
\NormalTok{sample\_props150 }\OtherTok{\textless{}{-}}\NormalTok{ global\_monitor }\SpecialCharTok{\%\textgreater{}\%}
                    \FunctionTok{rep\_sample\_n}\NormalTok{(}\AttributeTok{size =} \DecValTok{150}\NormalTok{, }\AttributeTok{reps =} \DecValTok{2000}\NormalTok{, }\AttributeTok{replace =} \ConstantTok{TRUE}\NormalTok{) }\SpecialCharTok{\%\textgreater{}\%}
                    \FunctionTok{count}\NormalTok{(scientist\_work) }\SpecialCharTok{\%\textgreater{}\%}
                    \FunctionTok{mutate}\NormalTok{(}\AttributeTok{p\_hat =}\NormalTok{ n }\SpecialCharTok{/}\FunctionTok{sum}\NormalTok{(n)) }\SpecialCharTok{\%\textgreater{}\%}
                    \FunctionTok{filter}\NormalTok{(scientist\_work }\SpecialCharTok{==} \StringTok{"Benefits"}\NormalTok{)}
\end{Highlighting}
\end{Shaded}

\begin{Shaded}
\begin{Highlighting}[]
\CommentTok{\# Create histogram of the sampling distribution}
\FunctionTok{ggplot}\NormalTok{(}\AttributeTok{data =}\NormalTok{ sample\_props150, }\FunctionTok{aes}\NormalTok{(}\AttributeTok{x =}\NormalTok{ p\_hat)) }\SpecialCharTok{+}
  \FunctionTok{geom\_histogram}\NormalTok{(}\AttributeTok{binwidth =} \FloatTok{0.02}\NormalTok{) }\SpecialCharTok{+}
  \FunctionTok{labs}\NormalTok{(}
    \AttributeTok{x =} \StringTok{"p\_hat (Benefits)"}\NormalTok{,}
    \AttributeTok{title =} \StringTok{"Sampling distribution of p\_hat"}\NormalTok{,}
    \AttributeTok{subtitle =} \StringTok{"Sample size = 150, Number of samples = 2000"}
\NormalTok{  )}
\end{Highlighting}
\end{Shaded}

\includegraphics{Part1_sampling_distributions-EKO_files/figure-latex/unnamed-chunk-4-1.pdf}

\textbf{Shape comparison:}

\textbf{Current distribution (n=150):}

\textbf{Much more symmetrical and clearly bell-shaped More concentrated
around the center (around 0.80) Narrower spread, ranging roughly from
0.70 to 0.90 Very close to a normal distribution}

\textbf{Previous distribution (n=15):}

\textbf{Was less symmetrical and showed some left skew Had more
variability and spread Ranged more widely from about 0.30 to 0.90 Was
more irregular in shape}

\textbf{Best guess for true proportion: Based on this sampling
distribution with n=150, I would estimate the true population proportion
to be about 0.80 (80\%) because:This is where the distribution centers
and peaks .The distribution is very symmetrical around this value With
the larger sample size of 150, we have more precision in our estimate
than with n=15. This increased precision is evident in the narrower
spread of the sampling distribution}

\textbf{This improved shape and precision illustrates why larger sample
sizes give us more reliable estimates of population parameters.}

\begin{enumerate}
\def\labelenumi{\arabic{enumi}.}
\setcounter{enumi}{9}
\tightlist
\item
  Of the sampling distributions from 2 and 3, which has a smaller
  spread? If you're concerned with making estimates that are more often
  close to the true value, would you prefer a sampling distribution with
  a large or small spread?\\
\end{enumerate}

The sampling distribution with n=150 has a notably smaller spread than
the one with n=15. We can see this clearly by comparing the two
distributions:

n=150 distribution:Values mostly concentrated between 0.70 and 0.90.
Tight, compact shape. Clear peak around 0.80

n=15 distribution: Values spread widely from 0.30 to 0.90. Much more
dispersed shape. Less concentrated around the center

When making estimates, you would prefer a sampling distribution with a
smaller spread because: Sample proportions will be more consistently
close to the true population value. Less variability means more precise
estimates. Higher confidence that any single sample will give you a
value close to the true population proportion

This is why larger sample sizes (like n=150) are generally preferred -
they give us more precise estimates by reducing the variability in our
sampling distribution. With the smaller sample size (n=15), there's a
much higher chance of getting a sample proportion that's far from the
true population value.

\begin{center}\rule{0.5\linewidth}{0.5pt}\end{center}

\end{document}
